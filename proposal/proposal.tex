\documentclass[]{scrartcl}
\usepackage[style=alphabetic]{biblatex}
\addbibresource{bibliography.bib}
%opening
\title{Proposal Computational Psychology}
\begin{document}
\def\isblind{1}
\if\isblind0
    \author{Jason Cramer, Maximilian Dierschke, Nils Engleder}\fi
\if \isblind1
    \author{}\fi
\maketitle
\section{Background}
The problem are project is focusing on is Shepard's ideal generalization problem. The generalization problem focuses on how humans build hypothesis spaces for a given consequence after observing stimuli.
In the paper "Generalization, similarity, and bayesian inference", they discuss how using a model of bayesian inference, we can predict the probabilty of given stimuli being included within the consequential region \cite{Tenenbaum}.
The model uses the equation $p(y \in C \mid x) = \sum\limits_{h:y\in h} p(h | x)$ where $h$ is a hypothesis from the hypothesis space ${\cal H}$ and $p(h | x)$ is the posterior probabilty of  the hypothesis after observing x.
We plan to extend this model to investigate how including negative examples within the x vector(x is the observed stimuli) affect how the model limits hypotheses. We also plan to explore how different distrubitions and models compare to the original model for generalization.
\section{Question}
How can the model be improved to handle negative examples and discontinuous hypotheses?
\section{Method}

\nocite{*}
\printbibliography
\end{document}
