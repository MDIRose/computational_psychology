\documentclass[]{scrartcl}
\usepackage[style=alphabetic]{biblatex}
\usepackage[shortlabels]{enumitem}
\addbibresource{bibliography.bib}
%opening
\title{Proposal Computational Psychology}
\begin{document}
\def\isblind{1}
\if\isblind0
    \author{Jason Cramer, Maximilian Dierschke, Nils Engleder}\fi
\if \isblind1
    \author{}\fi
\maketitle
\section{Background}
The problem are project is focusing on is Shepard's ideal generalization problem. The generalization problem focuses on how humans build hypothesis spaces for a given consequence after observing stimuli.
In the paper "Generalization, similarity, and bayesian inference", they discuss how using a model of bayesian inference, we can predict the probabilty of given stimuli being included within the consequential region \cite{Tenenbaum}.
The model uses the equation $p(y \in C \mid x) = \sum\limits_{h:y\in h} p(h | x)$ where $h$ is a hypothesis from the hypothesis space ${\cal H}$ and $p(h | x)$ is the posterior probabilty of  the hypothesis after observing x.
We plan to extend this model to investigate how including negative examples within the x vector(x is the observed stimuli) affect how the model limits hypotheses. We also plan to explore how different distrubitions and models compare to the original model for generalization.
\section{Question}
How can the model be improved to handle negative examples and discontinuous hypotheses?
\section{Method}
\begin{enumerate}[(1)]
	\item Evaluate how negative examples affect the predictions of the original model and use it as a baseline
	\item Evaluate the predictions made by the original model for multi cluster sample data 
	\item Evaluate how the model could be altered so that negative examples can be incorporated to limit cluster intervals assuming a continuous consequential region
	\item Evaluate how the model could be altered so that negative examples can be incorporated to limit cluster intervals assuming a discontinuous consequential region
	\item Evaluate how the model can be expanded to handle discontinuous regions without negative examples
\end{enumerate} 
We want to initially solve the first three issues. Due to the added complexity we then have to decide based on the results how we limit the scope of the rest of the project for (4) and especially (5).


\nocite{*}
\printbibliography
\end{document}
